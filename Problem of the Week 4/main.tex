\documentclass[12pt]{article}

\usepackage{parskip}
\usepackage[margin=1in]{geometry} 
\usepackage{amsmath,amsthm,amssymb}
\usepackage{listings}
\usepackage{color}

\definecolor{dkgreen}{rgb}{0,0.6,0}
\definecolor{gray}{rgb}{0.5,0.5,0.5}
\definecolor{mauve}{rgb}{0.58,0,0.82}

\lstset{frame=tb,
  language=Python,
  aboveskip=3mm,
  belowskip=3mm,
  showstringspaces=false,
  columns=flexible,
  basicstyle={\small\ttfamily},
  numbers=none,
  numberstyle=\tiny\color{gray},
  keywordstyle=\color{blue},
  commentstyle=\color{dkgreen},
  stringstyle=\color{mauve},
  breaklines=true,
  breakatwhitespace=true,
  tabsize=3
}
 
\newcommand{\N}{\mathbb{N}}
\newcommand{\Z}{\mathbb{Z}}
 
\newenvironment{theorem}[2][Theorem]{\begin{trivlist}
\item[\hskip \labelsep {\bfseries #1}\hskip \labelsep {\bfseries #2.}]}{\end{trivlist}}
\newenvironment{lemma}[2][Lemma]{\begin{trivlist}
\item[\hskip \labelsep {\bfseries #1}\hskip \labelsep {\bfseries #2.}]}{\end{trivlist}}
\newenvironment{exercise}[2][Exercise]{\begin{trivlist}
\item[\hskip \labelsep {\bfseries #1}\hskip \labelsep {\bfseries #2.}]}{\end{trivlist}}
\newenvironment{reflection}[2][Reflection]{\begin{trivlist}
\item[\hskip \labelsep {\bfseries #1}\hskip \labelsep {\bfseries #2.}]}{\end{trivlist}}
\newenvironment{proposition}[2][Proposition]{\begin{trivlist}
\item[\hskip \labelsep {\bfseries #1}\hskip \labelsep {\bfseries #2.}]}{\end{trivlist}}
\newenvironment{corollary}[2][Corollary]{\begin{trivlist}
\item[\hskip \labelsep {\bfseries #1}\hskip \labelsep {\bfseries #2.}]}{\end{trivlist}}
 
\begin{document}
 
% --------------------------------------------------------------
%                         Start here
% --------------------------------------------------------------
 
%\renewcommand{\qedsymbol}{\filledbox}
 
\title{Problem of the Week 4}%replace X with the appropriate number
\author{JFB} %if necessary, replace with your course title
 
\maketitle

By the requirements of the problem see that Peter's final number must be $666$, $669$, $696$, $699$,  $966$, $999$, $996$, or $969$.

Writing a program, we can brute force check that the Peter's original number could be any element of 
$$
\{135, 138, 234, 237, 333, 336, 432, 435, 531, 534, 630, 633, 732, 831, 930 \}
$$
One can check each of these solutions and see that any one of them may have been Peter's number.

Here is the program that verifies this solution,

\begin{lstlisting}
targets = [666, 669, 696, 699, 999, 966, 996, 969] 
three_digit_nums = list(range(100,1000))

good_nums = []
for num in three_digit_nums:
    string_num = str(num)
    reverse_num = int(string_num[::-1])
    if reverse_num + num in targets:
        good_nums.append(num)
print(good_nums)
[135, 138, 234, 237, 333, 336, 432, 435, 531, 534, 630, 633, 732, 831, 930]
\end{lstlisting}


\end{document}
